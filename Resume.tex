\documentclass[10pt,a4paper]{article}
\usepackage[utf8]{inputenc}
\usepackage[italian]{babel}
\usepackage{amsmath}
\usepackage{amsfonts}
\usepackage{amssymb}
\usepackage{subcaption}
\usepackage[separate-uncertainty=true]{siunitx}
\usepackage[left=2cm,right=2cm,top=2cm,bottom=2cm]{geometry}
\newcommand{\rem}[1]{[\emph{#1}]}
\newcommand{\exn}{\phantom{xxx}}
\renewcommand{\thesubsection}{\thesection.\alph{subsection}}  %% use 1.a numbering
\usepackage{verbatim}
\usepackage[title]{appendix}
\usepackage{mathtools}
\usepackage{bm}
\usepackage[super]{nth}
\usepackage{bbold}

\usepackage{epsfig,scrpage2,graphicx}

\setcounter{secnumdepth}{3}

\setlength{\parindent}{0em}
\setlength{\parskip}{0ex plus0.5ex minus0ex}
\pagestyle{scrheadings}
\bibliographystyle{unsrt}   

\renewcommand{\headfont}{\normalfont}

\cfoot{\pagemark}

\setlength\parindent{0pt}
\author{
	Edoardo Centamori*\\
	\newline
	Scuola Normale Superiore\\
	\newline
	Università di Pisa
}
\title{
	\hspace{15pt}\includegraphics[scale=0.28]{DESYlogo.pdf}
	\hspace{20pt}\includegraphics[scale=0.15]{SNS.jpg}
	\hspace{50pt}\includegraphics[scale=0.15]{uni-pisa.jpg}
	\newline
	\huge Summer Student Project}

%ADD SUPERVISORS
\begin{document}

\date{\today}
\maketitle
\section{Abstract}
 WIP.
\section{Theoretical introduction}

\subsection{PIMD}
It's well known that path integral reformulation of quantum mechanics yield to 
\[K(x',t,x,0)=\int [\mathcal{D}x(\tau)]e^{-S[x(\tau)]/\hbar} \]
where K is the propagator, $S[x(\tau)]$ the action of the path $x(\tau)$ and $[\mathcal{D}x(\tau)]$ indicates the set of path starting from $x$ at time $0$ and getting to $x'$ at time $t$.
%ACTUALLY NOT SURE ABOUT X AND X', WHICH IS WHICH?

Then considering a simple minkowskian rotation, we can obtain a partition function

\[\mathcal{Z}(\beta)=\oint [\mathcal{D}x(\tau)]e^{iS[x(\tau)]/\hbar} \]

where now integration is intended over all closed path of imaginary time $\beta\hbar$.

Since we would like to compute this quantities numerically we need to discretize such equation, in particular is straightforward to show that
\[ \mathcal{Z}_N(\beta):= \left( \frac{mN}{2\pi \beta \hbar^2}  \right)^{N/2} \int dx_1\cdots dx_N \exp \left\{ -\sum_{i=1}^{N}\frac{mN}{2\beta\hbar^2}(x_{i+1}-x_i)^2+\frac{\beta}{N}V(x_i)  \right\} \xrightarrow[N\rightarrow \infty]{} \mathcal{Z}(\beta)\]

where the indexes are intended to be cyclic.

What is particularly surprising is that 

\[ \left( \frac{mN}{2\pi \beta \hbar^2}  \right)^{N/2} = \int \frac{dp_1\cdots dp_N}{\hbar^N} \exp\left\{ -\beta\sum_{i=1}^{N}\frac{ p_i^2}{2m} \right\} \]

So that 
\[\mathcal{Z}_N(\beta)= \int \frac{dx_1\cdots dx_Ndp_1\cdots dp_N}{\hbar^N} e^{-\beta_N H_n}   \]
\[ H_n(x_1,\dots,x_N,p_1,\dots,p_N)= \sum_{i=1}^{N}\frac{ p_i^2}{2m} + \frac{m\omega_N^2}{2}(x_{i+1}-x_i)^2+V(x_i)\]
with $\omega_N=\frac{N}{\beta\hbar}$ and $\beta_n=\frac{\beta}{N}$.
but this is actually the partition function of a classical system, more specifically it's the partition function of a set of $N$ beads cyclically connected by harmonic springs with frequency $\omega_N$ and immersed in a potential $V$, the system is canonically distributed with inverse temperature $\beta_N$.
\vspace{15pt}

This idea provide us a method for simulating quantum system using classical algorithm, in particular $\mathcal{Z}_N(\beta)$ could be evaluated using either a Monte Carlo sampling or a molecular dynamics simulation, we focus on the latter option.
Certain problem arise from this task, the first one is ergodicity, in order to achieve accurate averages along the canonical distribution we need the system to be actually canonically distributed, this require the presence of a thermostat, the most used are certainly the Nosé-Hoover and the Langevin thermostats, we'll describe the second one.
%CITATION

\subsection{SDE and Langevin thermostating}
Langevin equations
\[\frac{dq}{dt}=\frac{p}{m} \]
\[\frac{dp}{dt}=-\frac{dV}{dq}-\gamma p+\sqrt{\frac{2m\gamma}{\beta}} \xi \]
Is a set of stochastic differential equation, $\gamma$ is a viscous coefficient and $\xi$ a random variable with $\langle\xi(t)\rangle=0$ $\langle\xi(t)\xi(s)\rangle = \delta(t-s)$ .
By means of Ito calculus it's fairly easy to transform such equation in an equation concerning the density distribution of the random variable $p$, such equation is known as Fokker-Plank equation
\[ \frac{\partial\rho}{\partial t} = -p\frac{\partial\rho}{\partial q}+\frac{\partial}{\partial p}[(\frac{dV}{dq}+\gamma p)\rho]+\frac{m\gamma}{\beta}\frac{\partial^2\rho}{\partial p^2} \]

where $\rho= \rho(q,p,t|q_0,p_0,0)$ is a conditional distribution for the stochastic variable $(q,p)$ in the phase space.
By direct substitution it's easy to check that
\[\rho = C \exp	\left\{ -\beta(\frac{p^2}{2m}+V(q)) \right\}  \]
with C fixed by the normalization condition is the only stationary solution.
We can workout analytically a solution for the SDE even in the time dependent case, we simply need a little bit of Ito calculus. 

Precisely for the case in which $V$ is an harmonic potential then the Langevin equation actually reduce to a simple Ornstein-Uhlenbeck process, for which the general SDE is:
\[dx = -kxdt+\sqrt{D} dW(t) \] 
Where $dW(t)$ is the differential of the Wigner process.
Considering the stochastic variable $y=xe^{kt}$ and using Ito formula yield a very simple SDE 
\[dy = \sqrt{D} E^{kt} dW(t) \] 
that can be easily integrated so that
\[x(t) = x(0)e^{-kt}+\sqrt{D} \int_0^t e^{-k(t-t')}dW(t') \]
which, if $x(0)$ is either gaussian or nonrandom is clearly a gaussian process, thus completely specified by mean and variance (which can be evaluated easily using Ito's formulas for averages an correlations):
\[\langle x(t)\rangle =  \langle x(0)\rangle e^{-kt} \]
\[ \text{Var} \{x(t) \}= \{\text{Var}\{x(0)\}  -\frac{D}{2k} \}e^{-2kt} +\frac{D}{2k} \]
The exponential decaying behavior shows that not only the stationary solution of this process is the canonical distribution but also that any process, regardless of it's initial condition approaches the stationary solution with a time scale dictated by the viscous coefficient. In future sections we'll search for optimal $\gamma$ coefficient in order to find efficient integrators of the motion.
The relevant concept for the moment is that such equation of motion effectively act as a thermostat imposing canonical distribution to the paths. 

\subsection{Rattle algorithm}

Sometimes in these simulations is important to impose certain constrains, for example if I'm simulating the scattering of a set of particles I might want to ignore the energy associated to the motion of the center of mass of the system and thus I might want to remove such degree of freedom from my simulation. 

For classical system this is done by mean of the \emph{Rattle algorithm}.
Basically you would like to evaluate the generalized forces coming from the constrain, and for time independent constrain of the form :
\[g_i(\vec{q}) = 0  \text{ for } i=1,\dots,m \]
such forces must be perpendicular to the surface defined by the constrain itself (because of D'Alembert principle). Which in the end yield to :

\[
\begin{cases} 
	\frac{d\vec{q}}{dt} = \vec{v} \\ 
	\hat{M} \frac{d\vec{v}}{dt} = - \vec{\nabla}_{q}V(\vec{q})- \sum_i^m \lambda_i  \vec{\nabla}_{q}g_i(\vec{q})\\ 
	g_i(\vec{q})=0
\end{cases}
\]
Where the $\lambda_i$ can be thought also as Lagrange multipliers, and can be found explicitly by imposing the constrain (third equation).
Rattle algorithm simply reimpose such constrains at each steps as will be shown more in detail in subsequent sections.
\section{Computational implementation}
\subsection{Classical Stormer-Verlet integrator}
I started implementing a classical integrator to simulate the motion of necklaces immersed in an harmonic potential (basically PIMD without thermalization).
The integrator chosen is the Stormer-Verlet algorithm, which is a second order integrator (meaning that at each step I commit and error that goes as $o(dt^3)$ with $dt$ time step in the discretization). Specifically the implemented algorithm at each steps act as :
\begin{enumerate}
	\item \[p \leftarrow p - \frac{dV}{d\vec{x}}\frac{dt}{2}\]
	\item \[ q \leftarrow q + \frac{p}{m}\frac{dt}{2}\]
	\item \[ p \leftarrow p - \frac{dV}{d\vec{x}}\frac{dt}{2}\]
\end{enumerate}
In order to check the effective functionality of such integrator I checked for conservation of classical energy and angular momentum of the system (considered completely classical at the moment) which are reported in Fig \ref{fig: e_m}.
\begin{figure}[h]
	\begin{center}
		\includegraphics[width=0.5\linewidth]{imm_rep/classical/e_m.png}
	\end{center}
	\caption{Energy and angular momentum conservation for the classical Verlet-Storm algorithm.}
	\label{fig: e_m}
\end{figure}
As can be seen explicitly, energy is conserved up to one part in a million while angular momentum is conserved up to floating point precision for this specific case. In general such fluctuation on energy are due to the finite time step in the integration and can be reduced reducing the time step as the general fluctuation goes as
\[ \sqrt{(\Delta E^2)} = o(dt^2)  \]
(More precisely there is a theoretical bound that goes as $o(dt^2)$ ).

What is not obvious in general (and in general not true for all the commonly used integrators) is the conservation of the mean value of energy, which is true for this specific algorithm for $dt$ sufficiently small (while to big values of $dt$ can bring to systematic drift which have in general an exponentially divergent behavior as shown in Fig \ref{fig: exp_en} (notice how in a few dozen of steps the energy increases by over eight order of magnitudes). Such conservation is not obvious at all because in general small error at each steps accumulate so that trajectories diverges exponentially from the exact ones, nevertheless such exponentially wrong trajectory still has the correct energy for this particular integrator.


\begin{figure}[h]
	\begin{center}
		\includegraphics[width=0.5\linewidth]{imm_rep/classical/exp_en.png}
	\end{center}
	\caption{Divegence of energy for too big values of $dt$.}
	\label{fig: exp_en}
\end{figure}

\vspace{15pt}

To have a better visualization of what is happening I have also created an algorithm to visualize dynamically the motion of the classical system (for clear reasons the visualization is only implemented for 2-dimensional system and could be easily extended to 3-dimensional ones while the computing algorithm is fully generalized to any number of dimension). Some randomly chosen frames of the video generated by this algorithms are shown in Fig \ref{fig: classic}.
\begin{figure}[h]
	\begin{center}
		\includegraphics[width=0.5\linewidth]{imm_rep/classical/classic.png}
	\end{center}
	\caption{Some random frames of the classical evolution of 2 necklace with 10 beads each.}
	\label{fig: classic}
\end{figure}

\subsection{Normal modes simplification}
Notice that in this PIMD evolution, the hamiltonian can be decomposed as
\[ H_n(x_1,\dots,x_N,p_1,\dots,p_N)= \sum_{i=1}^{N}\frac{ p_i^2}{2m} + \frac{m\omega_N^2}{2}(x_{i+1}-x_i)^2+V(x_i) = H_n^0 + V_n \]
where $H_n^0$ is the hamiltonian associated with a free necklace evolution and $V_n= \sum_i V(x_i)$.
Then the following decomposition is called Trotter splitting
\[ e^{\mathcal{L}t} = \lim\limits_{P\rightarrow \infty} (e^{\mathcal{L}^0\frac{t}{P}} e^{\mathcal{L}_V\frac{t}{P}})^{P} \]
where $\mathcal{L}=\mathcal{L}^0+\mathcal{L}_V$ are the liouvillian associated to $H_n = H_n^0 + V_n$.
So we can actually evolve separately the free necklace before and then the free replicas (ignoring the harmonic springs connecting them) inside the potential $V_n$ 
at each step. Notice that Trotter splitting introduce an error in the propagation since we cannot really reach $P = \infty $ which would be the same as asking $dt=0$ in our algorithm, but it's important to know that the error introduced vanish as $dt \rightarrow 0$.
The advantage of such splitting is that the motion of the free necklace is that of a set of harmonic oscillators so we can analytically solve it.
More specifically, since in the free case the equation for the replicas are
\[\frac{d^2x_i}{dt^2}= -\omega_N^2(2x_i-x_{i+1}-x_{i-1})\]
then searching for 'discrete wave' normal modes of the form 
\[\tilde{x}_k=\sum_j e^{i\alpha_kj}x_j\]
\[x_k=\frac{1}{N}\sum_j e^{-i\alpha_kj}\tilde{x}_j\]
\[\tilde{x}_k= \tilde{x}_k^0 e^{-i\omega_kt}  \]
Direct substitution in the equation of motion yield
\[ \omega_k = 2\omega_N |\sin(\frac{\alpha_k}{2})| \]
%MAYBE ALPHA/2 CHECK
while requirement of cyclicity $x_{N}=x_0$ impose $\alpha_k = \frac{2\pi k}{N}$.
Transformation to normal modes is not really expensive since it can be done with a simple fast Fourier transform.
Anyway it should be noted that the eigenfrequencies are degenerate since $\omega_k = \omega_{N-k}$, so in general to avoid the usage of complex numbers in the algorithm linear combination of the degenerate $\tilde{x}$ are chosen so that (assume N even)

 \[
C_{jk} = \left\{\begin{array}{lr}
\sqrt{\frac{1}{N}}, & k=0\\
\sqrt{\frac{2}{N}} \cos(2\pi jk/N), & 1\leq k\leq\frac{N}{2}-1\\
\sqrt{\frac{1}{N}} (-1)^j, &  k= \frac{N}{2} \\
\sqrt{\frac{2}{N}} \sin(2\pi jk/N), &  \frac{N}{2}+1\leq k \leq N
\end{array}\right.\]

\[\tilde{q}_k = \sum_j C_{jk}q_j  \]
\[\tilde{p}_k = \sum_j C_{jk}p_j  \]
\[q_j = \sum_j C_{jk}\tilde{q}_k  \]
\[p_j = \sum_j C_{jk}\tilde{p}_k  \]
Since now the normal modes represent independent harmonic oscillator we can easily solve their motion, thanks to this consideration our improved version of the Stormer-Verlet algorithm is simply
\begin{enumerate}
	\item \[p_k \leftarrow p_k - \frac{dV_n}{d\vec{x_k}}\frac{dt}{2}\]
	\item 
	\begin{equation}
		\begin{pmatrix} 
	\tilde{p}_k  \\
	\tilde{q}_k \\
	\end{pmatrix} 
	\leftarrow
	\begin{pmatrix} 
	\cos(\omega_k dt) & -m\omega_k \sin(\omega_k dt)   \\
	\frac{1}{m\omega_k}\sin(\omega_k dt) & \cos(\omega_k)  \\
	\end{pmatrix} 
	\begin{pmatrix} 
	\tilde{p}_k  \\
	\tilde{q}_k \\
	\end{pmatrix} 
	\end{equation}
	\item \[ p_k \leftarrow p_k - \frac{dV_n}{d\vec{x}_k}\frac{dt}{2}\]
\end{enumerate}
This idea is particularly useful if the frequency $\omega_N$ is the biggest in the system (which is always the case since $\omega_N \propto N$) since this evolution is exact, so no error is introduced in the evolution of the free necklace and bigger $dt$ can be used without affecting the results significantly.
\subsection{Classical Rattle implementation}
As we said previously Rattle algorithm at each step assure the preservation of constrain, and as can be seen is a direct generalization of the Stormer-Verlet algorithm
\[
\begin{cases}
\bm{q}^{n+1} =  \bm{q}^n + dt\bm{v}^{n+\frac{1}{2}}\\
\bm{M}\bm{v}^{n+\frac{1}{2}} = \bm{M}\bm{v}^n-\frac{dt}{2}\bm{\nabla}_{\bm q}V(\bm{q}^n)-\frac{dt}{2} \bm{G}(\bm{q}^n)^T\bm{\lambda}_r^n\\
\bm{g}(\bm{q}^{n+1}) = \bm{0}\\
\bm{M}\bm{v}^{n+1} = \bm{M}\bm{v}^{n+\frac{1}{2}}-\frac{dt}{2}\bm{\nabla}_{\bm q}V(\bm{q}^{n+1})-\frac{dt}{2} \bm{G}(\bm{q}^{n+1})^T\bm{\lambda}_v^{n+1}\\
\bm{G}(\bm{q}^{n+1})\bm{v}^{n+1} = \bm{0}
\end{cases}
\]
where $\bm{G}^T(\bm{q})=\bm{\nabla}_{\bm q}\bm{g}(\bm{q})$. The two constrains are the requirement of $\bm{q}$ to point on the constrained surface and of $\bm{v}$ to lie on it.
In practice the algorithm ar each step act as:
\begin{enumerate}
	\item Evaluate $\bm{\lambda}_r$ by solving :
	\[ \bm{g}(\bm{q} + dt\bm{v}-\frac{dt^2}{2}\bm{M}^{-1}(\bm{\nabla}_{\bm q}V(\bm{q})-\bm{G}(\bm{q})^T\bm{\lambda}_r)) = \bm{0} \]
	\item 
	\[
	\bm{v} \leftarrow \bm{v}-\frac{dt}{2}\bm{M}^{-1}\bm{\nabla}_{\bm q}V(\bm{q})-\frac{dt}{2} \bm{M}^{-1}\bm{G}(\bm{q})^T\bm{\lambda}_r
	\]
	\item
	\[ \bm{q} \leftarrow  \bm{q} + dt\bm{v} \]
	\item Evaluate $\bm{\lambda}_v$ by solving :
	\[
	\bm{G}(\bm{q})(\bm{v}-\frac{dt}{2}\bm{M}^{-1}\bm{\nabla}_{\bm q}V(\bm{q})-\frac{dt}{2}\bm{M}^{-1} \bm{G}(\bm{q})^T\bm{\lambda}_v) = \bm{0}
	\]
	\item
	\[
	\bm{v} \leftarrow \bm{v}-\frac{dt}{2}\bm{M}^{-1}\bm{\nabla}_{\bm q}V(\bm{q})-\frac{dt}{2} \bm{M}^{-1}\bm{G}(\bm{q})^T\bm{\lambda}_v
	\]
\end{enumerate}
%ADD ALSO IN COMBINATION WITH NORMAL MODE TRICK
Solving the implicit equations to evaluate $\bm{\lambda}_r$ in the most general case will require the utilization of Newton method at each step which would probably become the most expensive part of the algorithm, fortunately for linear or quadratic constrains (and technically also cubic and quartic, but I haven't implement them explicitly) such equation can be solved analytically thus removing this burden (notice that since $\nth{5}$ grade or more equations cannot be solved explicitly such idea cannot be further generalized).
It's important to say that Rattle is still a second order integrator.
\newline

Trying to simply add the normal mode trick is not entirely trivial since analytical equation get a little messy, for example considering the exact evolution of the free necklace as we've done previously but adding the constraining force yield to
\[ \bm{q}_i^{n+1}=C_{ik}C_{kj}[\cos(\omega_kdt)\bm{q}_j+\frac{1}{\omega_k}\sin(\omega_kdt)(\bm{v}_j^n-\frac{dt}{2}\bm{M}^{-1} \bm{\nabla}_{\bm{q}}V_n(\bm{q}_j) -\bm{M}^{-1} \bm{G}(\bm{q})^T\bm{\lambda}_r )]  \] 
So that in theory one could solve the implicit equation $\bm{g}(\bm{q}^{n+1}) = \bm{0}$ analytically for the constrain and find the corresponding Lagrange multipliers. What I found out is that this is not always necessary and can be implemented only if more efficiency is required, in general since Stormer-Verlet algorithm is an approximation of the version with the normal mode trick, and it's true that in the limit $dt \rightarrow 0$ the two algorithm become identical, we could think to use the value of $\bm{\lambda}_r$ and $\bm{\lambda}_v$ that you could find considering the standard algorithm and apply them in the new version, this work because the small error introduced does not propagate in time since the constrain are reimposed at each step. The effect of using this simplified version is that the conservation of the constrained quantity will have slightly higher fluctuation around the preserved value but that can be made smaller by diminishing the value of $dt$.
%IN FUTURE YOU MIGHT THINK OF USING THE CORRECT ONE TO SEE WHAT WHOULD APPEN

As archetype of the general linear and quadratic constrain I'll show explicitly how to evaluate $\bm{\lambda}_r$ in order to constrain a centroid or the distance between two centroids.
\subsubsection{Fixed centroid}
Let $\bm{q}_{a,i}$ indicate the position of the i-th replica of the a-th necklace, then the requirement of the a-th centroid to be fixed is given by
\[\bm{g}_1 (\bm{q})= \frac{\sum_i \bm{q}_{a,i}}{N} = 	\bm0 \]
Let's say we're in $3$ dimension just for sake of clarity, then since that vectorial constrain can be written as $3$ different constrain we're supposed to find $6$ Lagrange multiplier $\lambda_{rx}$, $\lambda_{ry}$, $\lambda_{rz}$,$\lambda_{vx}$, $\lambda_{vy}$, $\lambda_{vz}$ which we collectively call $\bm{\lambda}_r$ and $\bm{\lambda}_v$.

By direct evaluation $\bm{G}(\bm{q})^T\bm{\lambda}_r$ assume value $\bm{0}$ for all the replicas that are not part of the fixed centroid and assume value $\bm{\lambda}_r$ otherwise.

Thus we can rewrite the first constrain as
\[\bm{c}^n+ \bm{v}_c^n dt - \frac{dt^2}{2} \bm{M}^{-1} \left(\frac{\sum_i^N  \bm{\nabla}_{\bm q}V(\bm{q}_{a,i}^n)}{N}  + \bm{G}(\bm{q})^T\bm{\lambda}_r^n \right) = 0 \]
where we indicated with $\bm{c}$ and $\bm{v}_c$ position and velocity of the centroid, then direct evaluation of the Lagrange multiplier is straightforward (actually it's easier to evaluate directly $ \bm{M}^{-1} \bm{G}(\bm{q})^T\bm{\lambda}_r^n$ since that's what appear in the formula for updating velocities)

The constrain on velocities is 
\[ \bm{0} = \bm{v}_c^{n+1} = \bm{v}_c^{n+\frac{1}{2}}-\frac{dt}{2}\frac{\sum_i^N  \bm{M}^{-1}\bm{\nabla}_{\bm q}V(\bm{q}_{a,i}^{n+1})}{N}-\frac{dt}{2} \bm{M}^{-1}\bm{\lambda}_v^{n+1}\]
that is once again a trivial linear equation for $\bm{M}^{-1}\bm{\lambda}_v^{n+1}$.
The correctness of the algorithm is checked both qualitatively thanks to the video visualization (some frames are shown in Fig \ref{fig: rattle}) and quantitatively by controlling that the energy remain conserved and plotting explicitly the position of the fixed centroid, as shown in Fig \ref{subfig: en_rat} and Fig \ref{subfig: cp}

\begin{figure}[h]
	\begin{center}
		\includegraphics[width=0.5\linewidth]{imm_rep/rattle/rattle.png}
	\end{center}
	\caption{Some random frames of the classical evolution of 2 necklace, one of which has his centroid constrained.}
	\label{fig: rattle}
\end{figure}

\begin{figure}[h]
	\begin{center}
		\begin{subfigure}[b]{0.4\textwidth}
		\centering
		\includegraphics[width=0.9\linewidth]{imm_rep/rattle/en_fixc.png}
		\caption{Energy and angular momentum.}
		\label{subfig: en_rat}
	\end{subfigure}
	%
	\begin{subfigure}[b]{0.4\textwidth}
		\centering
		\includegraphics[width=0.9\linewidth]{imm_rep/rattle/fixc.png}
		\caption{Centroid position conservation.}
		\label{subfig: cp}
	\end{subfigure}
\caption{Rattle algorithm for fixed centroid results.}
\end{center}

	\label{fig: en_rat}
\end{figure}

Naturally, fixing the position in a place different than $\bm{0}$ is straightforward and simply account for changing $\bm{\lambda}_r$ formula by $\bm{c}^n \rightarrow \bm{c}^n-\bm{c}_0$.

\subsubsection{Fixed lenght between centroids}
By the same token, since the constrain in this case is
\[ g_1(\bm{q})=\frac{1}{2}(|\bm{c}_a^{n+1} -\bm{c}_b^{n+1}|^2-l^2)=0  \]
since it's only one constrain we'll have only two Lagrange multiplier $\lambda_r$ and $\lambda_v$.
By direct derivation it's clear that 
\[
(\bm{G}(\bm{q})\bm{\lambda}_r)_{c,i} = \left\{\begin{array}{lr}
\bm{0}, & c\neq a  \wedge c \neq a\\
(\bm{c}_a -\bm{c}_b)\bm{\lambda}_r, & c=a\\
(\bm{c}_b -\bm{c}_a)\bm{\lambda}_r, & c=b
\end{array}\right.\]

so that the first constrain is rewritten as
\begin{align*}
| (\bm{c}_a^n-\bm{c}_b^n)+dt(\bm{v}_{ca}^n-\bm{v}_{cb}^n) - \frac{dt^2}{2} \bm{M}^{-1}\sum_i^N\left(\frac{  \bm{\nabla}_{\bm q}V(\bm{q}_{a,i}^{n})-\bm{\nabla}_{\bm q}V(\bm{q}_{b,i}^{n})}{N}\right) - \frac{dt^2}{2} \bm{M}^{-1} \frac{(\bm{c}_a^n-\bm{c}_b^n)}{N}\lambda_r |^2 - l^2 :=& \\
|\bm{A}+\bm{B}\lambda_r|^2 - l^2 = \bm{A}^2-\bm{B}^2\lambda_r^2 +2\bm{A}\cdot\bm{B}\lambda_r -l^2 = &0 
\end{align*}
Which is just a second order equation for $\lambda_r$ with solutions
\[\lambda_r = \frac{-\bm{A}\cdot\bm{B} \pm \sqrt{(\bm{A}\cdot\bm{B})^2-\bm{B}^2(\bm{A}^2-l^2)} }{\bm{B}^2} \]
The appropriate one is chosen by selecting the one with smaller absolute value since for $dt=0$ there is no constraining force, and assuming $\bm{\lambda}_r(dt)$ as an analytical function of $dt$ then for small enough values od $dt$ such multipliers must be small too.
The second constrain is 
\begin{align*}
 \bm{0}=(\bm{c}_a^{n+1}-\bm{c}_b^{n+1})(\bm{v}_{ca}^{n+1}-\bm{v}_{cb}^{n+1})=\\ (\bm{c}_a^{n+1}-\bm{c}_b^{n+1}) [\bm{v}_{ca}^{n+\frac{1}{2}}-\bm{v}_{cb}^{n+\frac{1}{2}}-\frac{dt}{2}\bm{M}^{-1}\left(\frac{\sum_i \bm{\nabla}_{\bm q}V(\bm{q}_{a,i}^{n})-\bm{\nabla}_{\bm q}V(\bm{q}_{b,i}^{n})}{N}\right)+(\bm{c}_a^{n+1}-\bm{c}_b^{n+1})\lambda_v]
\end{align*}
%CHECK FOR POSSIBLE FACTORS 2
Which is simply a linear equation and can be easily solved
%PUT SOLUTION
The algorithm has been implemented and results concerning conservation of the proper quantities are shown in Fig \ref{subfig: en_lenfc} and Fig \ref{subfig: lc}, while some random frames in order to visualize the motion of the classical system can be seen in Fig \ref{fig: lenfc1}



\begin{figure}[h]
	\begin{center}
		\begin{subfigure}[b]{0.4\textwidth}
			\centering
			\includegraphics[width=0.9\linewidth]{imm_rep/rattle/en_lenfc.png}
			\caption{Energy and angular momentum.}
			\label{subfig: en_lenfc}
		\end{subfigure}
		%
		\begin{subfigure}[b]{0.4\textwidth}
			\centering
			\includegraphics[width=0.9\linewidth]{imm_rep/rattle/lenfc.png}
			\caption{Length between centroid conservation.}
			\label{subfig: lc}
		\end{subfigure}
		\caption{Rattle algorithm for fixed length between centroids results.}
	\end{center}
	\label{fig: lenfc}
\end{figure}

\begin{figure}[h]
	\begin{center}
		\includegraphics[width=0.5\linewidth]{imm_rep/rattle/lenfc/lenfc.png}
	\end{center}
	\caption{Some random frames of the classical evolution of 2 necklace with fixed distance between centroids.}
	\label{fig: lenfc1}
\end{figure}


%PUT IMMAGES
\subsubsection{Combining constrains}
I would like to stress out that combining constrains is actually straightforward since the lagrange multiplier plays the role of the constraining forces and thus can be simply added all together thanks to superposition principle, meaning that we don't need to implement more complex algorithm, the one developed up to now already work together.
%PUT IMMAGES
\subsection{Thermostatting and PIMD implementation}
Let's ignore for the moment the Rattle implementation and focus on the Stormer-Verlet algorithm, if we want to use it to have a PIMD working algorithm we must add a thermostat to guarantee canonical distribution. As we saw, for white noise we have the Fokker-Plank equation
\[ \frac{\partial\rho}{\partial t} = -p\frac{\partial\rho}{\partial q}+\frac{\partial}{\partial p}[(\frac{dV}{dq}+\gamma p)\rho]+\frac{m\gamma}{\beta}\frac{\partial^2\rho}{\partial p^2} = -\mathcal{L}\rho\] 
Where $\mathcal{L}$ is the liouvillian operator, the fundamental idea is too introduce another Trotter splitting, $\mathcal{L}=\mathcal{L}_\gamma+\mathcal{L}_0$ with 
\[ \mathcal{L}_\gamma = -\gamma(\frac{\partial}{\partial p}p + \frac{m}{\beta}\frac{\partial^2}{\partial p^2})   \]
\[ \mathcal{L}_0 = \frac{p}{m}\frac{\partial}{\partial q}-\frac{dV(q)}{dq} \frac{\partial}{\partial p}  \]
So that $ \mathcal{L}_0$ is associated with the motion without thermostat and we already know how to treat with the Stormer-Verlet algorithm.
$ \mathcal{L}_\gamma$ can be integrated esplicitly too (in the sense of stochastic integration) and finally the algorithm add up to 

\begin{enumerate}
	\item \[\tilde{p}_k \leftarrow c_{1k}\tilde{p}_k + \sqrt{\frac{m}{\beta}}c_{2k} \xi_k \]
	\item \[p_k \leftarrow p_k - \frac{dV_n}{d\vec{x_k}}\frac{dt}{2}\]
	\item 
	\begin{equation}
	\begin{pmatrix} 
	\tilde{p}_k  \\
	\tilde{q}_k \\
	\end{pmatrix} 
	\leftarrow
	\begin{pmatrix} 
	\cos(\omega_k dt) & -m\omega_k \sin(\omega_k dt)   \\
	\frac{1}{m\omega_k}\sin(\omega_k dt) & \cos(\omega_k)  \\
	\end{pmatrix} 
	\begin{pmatrix} 
	\tilde{p}_k  \\
	\tilde{q}_k \\
	\end{pmatrix} 
	\end{equation}
	\item \[ p_k \leftarrow p_k - \frac{dV_n}{d\vec{x}_k}\frac{dt}{2}\]
	\item \[\tilde{p}_k \leftarrow c_{1k}\tilde{p}_k + \sqrt{\frac{m}{\beta}}c_{2k} \xi'_k \]
\end{enumerate}
Where $c_{1k} = e^{-\gamma_k \frac{dt}{2}}$ and $c_{2k}=\sqrt{1-c_{1k}^2}$. Technically $\gamma_k$ could be chosen arbitrarily and the canonical distribution would came up anyway but this could lead to inefficient sampling. 
%UNDERTSTAND BETTER IF V NON HARMONIC CAN BREAK THIS THING
For harmonic oscillator with frequency $\omega_k$ it can be shown that $\gamma_k = 2\omega_k$ is the optimal choice (meaning that is the one that minimize autocorrellation time for the energy).
\newline

%WRITE BETTER
%ADD IMMAGES OF THE RESULTS
Functionality of the algorithm has been tested carefully checking explicitly that for each normal mode, canonical distribution $dP \propto e^{-\beta_N H}dq^ddp^d$ (with $d$ number of dimensions of the classical system) is actually achieved.
More specifically this means that energies occurrences of individual normal modes will match Boltzmann distribution, more specifically, since in the case of harmonic external potential $V$ the modes are independent harmonic oscillator, we have
\[D(E)= \int \delta(E-E(\bm q,\bm p)) \frac{d\bm p d\bm p}{h^d} \propto E^{d-1}  \] 
\[ P(E) = D(E)e^{-\beta E}\propto E^{d-1}e^{-\beta E}  \]
that such distribution has actually been verified mode for mode by numerically fitting the canonical distribution (with parameter to fit $\beta_N$).
Fig \ref{fig: fit_beta} refer to the parameter $\beta=4. a.u.$ $N = 40$ $\beta_N=0.1 a.u.$. $\omega_{\text{ext}}$ is the frequency of the external harmonic oscillator.

\begin{figure}[h]
	\begin{center}
		\includegraphics[width=0.5\linewidth]{imm_rep/thermalization/fit_beta.png}
	\end{center}
	\caption{Fitted values of $\beta_N$ for every single mode.}
	\label{fig: fit_beta}
\end{figure}
 Fig \ref{subfig: 1d} and Fig \ref{subfig: 2d} refer to a specific mode fit (the \nth{5} chosen arbitrarily) and show explicitly the energies distribution in the case of $d=1,2$ (obtained binning all the energies achieved during a long time evolution in $20$ bins), the green curve represent the fitted function.
 
\begin{figure}[h]
	\begin{center}
		\begin{subfigure}[b]{0.4\textwidth}
			\centering
			\includegraphics[width=0.9\linewidth]{imm_rep/thermalization/1dfit.png}
			\caption{Case $d=1$.}
			\label{subfig: 1d}
		\end{subfigure}
		%
		\begin{subfigure}[b]{0.4\textwidth}
			\centering
			\includegraphics[width=0.9\linewidth]{imm_rep/thermalization/2dfit.png}
			\caption{Case $d=2$.}
			\label{subfig: 2d}
		\end{subfigure}
		\caption{Energies distribution.}
	\end{center}
	\label{fig: en_dis}
\end{figure}
 
 Once we have checked the correctness of such thermalization process we are basically sure that our PIMD algorithm is working correctly, thus since the external potential is the one of an harmonic oscillator we expect to find as energy of the system, fixed the temperature $\beta$ 
 \[\mathcal{Z}(\beta)= \sum_{n=0}^{+\infty} e^{-\beta\hbar\omega_{\text{ext}}(n+\frac{1}{2})} = \frac{e^{-\beta\frac{\hbar\omega_{\text{ext}}}{2}}}{1-e^{-\beta\hbar\omega_{\text{ext}}}}  \]
 \[E(\beta)= -\frac{d}{d\beta}\ln(\mathcal Z(\beta)) = \frac{\hbar\omega_{\text{ext}}}{2}\coth(\frac{\beta\hbar\omega_{\text{ext}}}{2}) \]
 We also need an estimator for the quantum energy of the system given his PIMD simulation, we'll use the primitive estimator. By computing 
 \[E_N = -\frac{d}{d\beta}\ln(\mathcal Z_N (\beta)) = \frac{\sum_{a,i} (\frac{1}{2}m_a v_{a,i}^2 - \frac{1}{2}m_i\omega_N^2(q_{a,i}-q_{a,i-1})^2+V_{\text{ext}}(q_{a,i}))}{N} \]
 that is called the \emph{primitive estimator}, we could also use the simplified version with 
 \[ \frac{\frac{1}{2}\sum_{a,i} m_a v_{a,i}^2}{N} \rightarrow \frac{Nd}{2\beta} \]
 Both of them are used to provide an estimate of the energy of the harmonic oscillator with $\omega_{\text{ext}} = 1 a.u.$  ($\hbar = 1 a.u.$) and $\beta= 4 a.u.$ in Fig \ref{fig: conv}
 \begin{figure}[h]
 	\begin{center}
 		\includegraphics[width=0.5\linewidth]{imm_rep/PIMD/buona.png}
 	\end{center}
 	\caption{Primitive estimators for the energy of a quantum harmonic oscillator.}
 	\label{fig: conv}
 \end{figure}
The process is repeated for several value of $\omega_{ext}$ and confronted with the expected value obtained by the previous theoretical calculation ($\beta$ is fixed and has value $\beta=4$), the results are shown in Fig \ref{fig: nice}.

 \begin{figure}[h]
	\begin{center}
		\includegraphics[width=0.5\linewidth]{imm_rep/PIMD/nice_fit.png}
	\end{center}
	\caption{Theoretical and computed values for the energy of a quantum harmonic oscillator.}
	\label{fig: nice}
\end{figure}
%ADD FIT
\subsection{White noise thermostat and Rattle combined}
Combining together Rattle algorithm and white noise Langevin thermalization yeld to a new step 

\begin{enumerate}
	\item \[\tilde{p}_k \leftarrow c_{1k}\tilde{p}_k + \sqrt{\frac{m}{\beta}}c_{2k} \xi_k \]
	\item \[\tilde{p}_k \leftarrow c_{1k}\tilde{p}_k + \sqrt{\frac{m}{\beta}}c_{2k} \xi'_k \]
	\item Evaluate $\bm{\lambda}_r$ by solving :
	\[ \bm{g}(\bm{q} + dt\bm{v}-\frac{dt^2}{2}\bm{M}^{-1}(\bm{\nabla}_{\bm q}V(\bm{q})-\bm{G}(\bm{q})^T\bm{\lambda}_r)) = \bm{0} \]
	\item \[p_k \leftarrow p_k - \frac{dV_n}{d\vec{x_k}}\frac{dt}{2}-(\bm{G}(\bm{q})^T\bm{\lambda}_r)_k\frac{dt}{2}\]
	\item 
	\begin{equation}
	\begin{pmatrix} 
	\tilde{p}_k  \\
	\tilde{q}_k \\
	\end{pmatrix} 
	\leftarrow
	\begin{pmatrix} 
	\cos(\omega_k dt) & -m\omega_k \sin(\omega_k dt)   \\
	\frac{1}{m\omega_k}\sin(\omega_k dt) & \cos(\omega_k)  \\
	\end{pmatrix} 
	\begin{pmatrix} 
	\tilde{p}_k  \\
	\tilde{q}_k \\
	\end{pmatrix} 
	\end{equation}
	\item Evaluate $\bm{\lambda}_v$ by solving :
	\[
	\bm{G}(\bm{q})(\bm{v}-\frac{dt}{2}\bm{M}^{-1}\bm{\nabla}_{\bm q}V(\bm{q})-\frac{dt}{2}\bm{M}^{-1} \bm{G}(\bm{q})^T\bm{\lambda}_v) = \bm{0}
	\]
	\item \[ p_k \leftarrow p_k - \frac{dV_n}{d\vec{x}_k}\frac{dt}{2}-(\bm{G}(\bm{q})^T\bm{\lambda}_v)_k\frac{dt}{2}\]
	
\end{enumerate}

There are a couple of things that need to be commented about the new step, first of all the expressions previously found for $\bm{\lambda}_r$ and $\bm{\lambda}_v$ must work at the same way in this algorithm simply replacing the previous $\bm{q}$ and $\bm{p}$ with the ones obtained after the thermalization since their utility is to simply reimpose the constrain at every step, then it should be noticed that the two thermalization steps coming from the Trotter splitting have been put together, this doesn't change anything since the step is repeated cyclically, but the relevant thing is that when we consider the final values of $\bm{q}$ and $\bm{p}$ at the end of ech cycle, they obeys the constrains (since no noise acted after they imposition), if I had not done such change the motion would have been the same at each step a part for a small noise coming from the second thermalization (that would have not cumulate in subsequent steps anyway tough).

The functionality of the algorithm has been checked by evaluating the effective constrained quantity over time as shown in Fig \ref{fig: mix} for a couple of necklace thermalized at $\beta=4$ and with the length between their centroid constrained.

 \begin{figure}[h]
	\begin{center}
		\includegraphics[width=0.5\linewidth]{imm_rep/PIMD_with_rattle/lenfc_therm.png}
	\end{center}
	\caption{Constrained length of the two centroids}
	\label{fig: mix}
\end{figure}

Some frames from the video visualization are shown in Fig \ref{fig: mixi}, there are some qualitative aspect that should be pointed out : 
\begin{itemize}
	\item[-] The motion of the replicas inside the necklace is completely unordered, that's because while in the previous example things have been initialized in a way to easier visualize the necklace shape, in this case the shape is rapidly lost due the ergodicity of the system
	\item[-] One of the necklace seem always more condensed than the other, that's just because in that specific simulation they have different masses ($m_1=4$ $a.u.$,$m_2=1$ $a.u.$) and by simple statistical mechanics calculation you expect their linear dimension to be proportional to the de Broglie wavelength of the particle that is $\Lambda=\frac{h}{\sqrt{2\pi mk_BT}}$
\end{itemize}

 \begin{figure}[h]
	\begin{center}
		\includegraphics[width=0.5\linewidth]{imm_rep/PIMD_with_rattle/mix.png}
	\end{center}
	\caption{Some frames of the themalization with rattle algorithm}
	\label{fig: mixi}
\end{figure}

Notice that since the constrain apply only to the centroids motion the other normal modes of the single centroids are unconstrained and must thus follow Boltzmann distribution as shown in Fig \ref{fig: mix_t} for a specific mode (the \nth{5} in this case) for $d=2$ .

 \begin{figure}[h]
	\begin{center}
		\includegraphics[width=0.5\linewidth]{imm_rep/PIMD_with_rattle/mix_t.png}
	\end{center}
	\caption{Boltzmann distribution for the energies of a specific mode in the new  algorithm.}
	\label{fig: mix_t}
\end{figure}
\subsection{A case of interest analysis}
My research group in the end want to use all this formalism in a ring polymer molecular dynamic (RPMD) simulation, which basically means that the motion of the necklace, which in PIMD was considered only a mathematical tool to evaluate averages for quantum operators, now is considered a real motion with identification of centroids and quantum particles, some justifications of this idea can be found in [REFERENCE].
%MAYBE YOU COULD WRITE THE MATZUBARA STAUFF ON YOUR OWN
A full investigation of this problem is far beyond the scope of my internship, but at a basic level we would like to be able to simulate molecules using the instruments seen up to now, rattle algorithm in particular will be used to constrain translational and rotational degree of freedom of such molecules that otherwise carry some energies that add up to the relevant results.
We'll work with the reaction 
\[Mu + H_2 \rightarrow MuH + H \]
%CHECK WHETER IS TRUE OR NOT
($Mu$ is muonium, an hydrogen atom in which proton has been replaced by a muon $\mu^+$), the reason behind this reaction is that muonium is particularly light with respect to normal atoms and thus quantum effect are much more marked and we would like to study the ability of this algorithm to take them in consideration, also since the system is particularly small it can be actually simulated by exact quantum mechanical scattering theory (which at the state of art cannot be performed computationally for system with more than 6 effective degree of freedom) an thus we can confront our results with the ones coming from a more sophisticated model. But anyway the algorithm developed will be fully functional for generic molecules.
Our approach will be to initialize the two system $Mu$ and $H_2$ so that they sample Boltzmann distribution for their real ab initio potential, since the potential are really complex and we cannot solve the problem analytically as we would do for an harmonic potential, the idea is the following:
\begin{itemize}
	\item[-] In the beginning approximate the potential connecting different atoms of the same molecules with an harmonic one (the one obtained by second order Taylor expansion of the real one) and ignore intermolecular forces by simply initializing different molecules far away
	\item[-] Use the thermostats seen previously to initialize the molecules in their ground state 
	\item[-] Change the potential slowly allowing the system to evolve naturally with the changing hamiltonian so that for the adiabatic theorem in the end the system will be initialized in the ground state of the real potential (the technique is called adiabatic switching)
	\item[-] Now that the system are correctly initialized a scattering process can be simulated by kicking the muonium in the hydrogen direction
\end{itemize}
So in the end we just need a couple of improvements in the previous steps seen, in particular the external potential must be replaced with a many-body one (harmonic in the beginning) so that the initial quantum system is described by
\[H=\sum_{I=1}^P \frac{p_I^2}{2m_I}+V(q_1, \dots,q_P)  \]
It easy to show that in this case the PIMD previously seen for monodimensional system generalize to the effective hamiltionian
\[H_n= \sum_{I=1}^P\sum_{i=1}^N \frac{(p_{I}^{(i)})^2}{2m_I} + \sum_{I=1}^P\sum_{i=1}^N\frac{m_i\omega_N^2}{2}(q_{I}^{(i)}-q_{I}^{(i+1)})^2 + \sum_i^N V(q_1^{(i)}, \dots,q_P^{(i)}) \]
%NOT COMPLETELY SURE ABOUT THE POTENTIAL, MAYBE THERE IS 1/N MULTIPLYING, I'LL HAVE TO CHECK
So that in general, considered two different atoms, only replicas with the same index $(i)$ interact with each other.
As I said previously the fixed centroid rattle algorithm is the archetype for generic linear constrain so generalization to the constrain of center of mass shall not be discussed since straightforward, results concerning the effective preservation of the constrain are shown in figure [REFERENCE]
%ADD REFERENCE 
%ADD BIBLIOGRAPHY
%PUT ENGLISH THEBIBLIOGRAPHY
\begin{thebibliography}{99}
	\begin{sloppypar}
		\bibitem{my_reference} Study of ...
		{\em Author name}
	\end{sloppypar}
\end{thebibliography}

\end{document}          
